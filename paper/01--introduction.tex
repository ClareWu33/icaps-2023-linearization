\section{Introduction}

Hierarchical Task Network (HTN) planning is a hierarchical approach to planning. As defined by \cite{HTNSurvey},
tasks in HTN planning are either primitive, corresponding to an action that can be taken, or compound. HTN problems have a set of methods that specify how one might achieve a given compound task, by decomposing it into a set of sub-tasks. A compound task may even decompose into itself, either directly via a method, or indirectly via a sequence of method applications. If decomposition leads to a sequence of primitive tasks executable from the initial state, then this sequence of actions is a solution to the problem, also known as a \emph{plan}.

In totally ordered HTN planning, or TOHTN planning, methods specify a total order on the sub-tasks. In partially ordered HTN planning, or POHTN planning, methods might only specify a partial order on the sub-tasks.

Many problems might be naturally more suited to being modelled as POHTN problems. However, for \emph{solving} the problem, it is desirable to have additional constraints that reduces the search space, while still preserving at least some of the actual solutions. For example, \cite{ErolHTNExpressivity} proved that POHTN planning is expressive enough to model undecidable problems, such as the language intersection problem of two context-free languages. On the other hand, TOHTN planning is not expressive enough to model undecideable problems. Where this is possible, it results in lower worst-case solving time. Converting the problem to a TOHTN problem allows us to exploit that fact to solve the problem more quickly.

Another benefit of transforming a POHTN problem to a TOHTN problem is that the additional structure to the problem could allow us to deploy specialised algorithms and heuristics for the totally ordered case.
The drawback to this approach is that, due to the greater expressivity of POHTN planning, there may exist POHTN problems that cannot be solved when converted to a TOHTN problem.
Fortunately, not every POHTN problem is guaranteed to be undecidable, and so could still be transformed while preserving at least one solution.
In this paper we present and investigate an algorithm for converting POHTN problems to TOHTN problems. We prove that when certain criteria are met, it guarantees that at least one solution will be preserved.
Also, we obtain a new class of decidable problems, namely those that satisfy the above mentioned criterion. Finally, we show that, even when these criteria are not met, very few problems are rendered unsolvable by the transformation, and that it greatly reduces solving time for problems, with gains being bigger for more difficult problems.
