\section{Future Work}

The maximum satisfiability problem (MAX-SAT) is the problem of determining the maximum number of clauses, of a given Boolean formula in conjunctive normal form, that can be made true by an assignment of truth values to the variables of the formula. MAX-SAT is used over SAT as it may not be possible to satisfy all orderings implied by the inferred preconditions and effects.

The current linearization procedure considers each method in isolation, and sub-tasks have their inferred $\PreS, \AddS, \DelS$. Thus we can treat each sub-task as a \emph{action}, and encode the method the same way as a classical plan, as described by \cite{RINTANEN201245}, but with the additional constraint that \emph{actions} chosen from the method sub-tasks can only be selected once, unlike in the domain.
