%%%% ijcai22.tex

\typeout{IJCAI--22 Instructions for Authors}

% These are the instructions for authors for IJCAI-22.

\documentclass{article}
\pdfpagewidth=8.5in
\pdfpageheight=11in
% The file ijcai22.sty is NOT the same as previous years'
\usepackage{ijcai22}

% Use the postscript times font!
\usepackage{times}
\usepackage{soul}
\usepackage{url}
\usepackage[hidelinks]{hyperref}
\usepackage[utf8]{inputenc}
\usepackage[small]{caption}
\usepackage{graphicx}
\usepackage{amsmath}
\usepackage{amsthm}
\usepackage{booktabs}
\usepackage{algorithm}
% \usepackage{algorithmic}
\urlstyle{same}

% the following package is optional:
%\usepackage{latexsym}

% See https://www.overleaf.com/learn/latex/theorems_and_proofs
% for a nice explanation of how to define new theorems, but keep
% in mind that the amsthm package is already included in this
% template and that you must *not* alter the styling.
\newtheorem{example}{Example}
\newtheorem{theorem}{Theorem}


% % % % % PACKAGES IMPORTED BY US % % % % %
\usepackage{amsfonts}
\usepackage{amsthm}
\usepackage{amsmath}
\usepackage{amssymb}
\usepackage{todonotes}
% \usepackage{mathbbol}
\usepackage[noend]{algpseudocode}
\usepackage{paralist}
\usepackage{booktabs}
\usepackage{xcolor}
\usepackage{subcaption}
\usepackage{tikz}
\usetikzlibrary{aiplans}


% Following comment is from ijcai97-submit.tex:
% The preparation of these files was supported by Schlumberger Palo Alto
% Research, AT\&T Bell Laboratories, and Morgan Kaufmann Publishers.
% Shirley Jowell, of Morgan Kaufmann Publishers, and Peter F.
% Patel-Schneider, of AT\&T Bell Laboratories collaborated on their
% preparation.

% These instructions can be modified and used in other conferences as long
% as credit to the authors and supporting agencies is retained, this notice
% is not changed, and further modification or reuse is not restricted.
% Neither Shirley Jowell nor Peter F. Patel-Schneider can be listed as
% contacts for providing assistance without their prior permission.

% To use for other conferences, change references to files and the
% conference appropriate and use other authors, contacts, publishers, and
% organizations.
% Also change the deadline and address for returning papers and the length and
% page charge instructions.
% Put where the files are available in the appropriate places.

% PDF Info Is REQUIRED.
% Please **do not** include Title and Author information
\pdfinfo{
/TemplateVersion (IJCAI.2022.0)
}

\title{``Order is Half of Life'' -- Establishing Totally Ordered HTN Planning Problems is Half the Job (to Getting them Solved Efficiently)}

\author{Anonymous}

% Single author syntax
% \author{
%     Author Name
%     \affiliations
%     Affiliation
%     \emails
%     pcchair@ijcai-22.org
% }

% Multiple author syntax (remove the single-author syntax above and the \iffalse ... \fi here)
\iffalse
\author{
First Author$^1$
\and
Second Author$^2$\and
Third Author$^{2,3}$\And
Fourth Author$^4$
\affiliations
$^1$First Affiliation\\
$^2$Second Affiliation\\
$^3$Third Affiliation\\
$^4$Fourth Affiliation
\emails
\{first, second\}@example.com,
third@other.example.com,
fourth@example.com
}
\fi





\begin{document}

\maketitle

\begin{abstract}
HTN planning is in general undecidable, yet there exist efficient planning algorithms. An important subclass is that of totally ordered HTN planning problems. Not only is this class decidable, but also easier to handle algorithmically since tasks cannot interleave anymore. Consequently, several specialized solvers exist, and dedicated tracks at the International Planning Competition (IPC) in both 2020 and 2023. We were able to show that almost all currently existing partially ordered planning benchmarks can be turned into totally ordered ones without sacrificing solvability. Establishing totally ordered problems in a preprocessing technique thus allows to use planning systems, heuristics, pruning techniques (etc.) that are only available for totally ordered problems. This allows solving problems more efficiently and with better-informed techniques, at the (little) cost of doing a preprocessing technique. We show that in only a very few cases linearization isn't possible, and in all others establishing a total order pays off \emph{significantly}, thus establishing our technique as a significant ingredient in the state of the art for solving partially ordered HTN problems. As a side contribution, we identify a new class of decidable (EXPTIME-complete) HTN problems, namely those where our preprocessing technique is guaranteed to be solvability-preserving.
\end{abstract}


% citiation commands
\newcommand{\shortCite}[1] {\citeauthor{#1} (\citeyear{#1})}

% symbols
\newcommand{\groundProb}{\ensuremath{\Pi}}
\newcommand{\lit}{\ensuremath{\mathcal{F}}}
\newcommand{\cTasks}{\ensuremath{\mathcal{C}}}
\newcommand{\pTasks}{\ensuremath{\mathcal{A}}}
\newcommand{\methods}{\ensuremath{\mathcal{M}}}
\newcommand{\domain}{\ensuremath{\mathcal{D}}}
\newcommand{\eff}{\ensuremath{\mathit{eff}}}
\newcommand{\effPos}[1]{\ensuremath{\eff{}^{+}(#1)}}
\newcommand{\effNeg}[1]{\ensuremath{\eff{}^{-}(#1)}}
\newcommand{\preCond}[1]{\ensuremath{\mathit{prec}(#1)}}
\newcommand{\effPosExt}[1]{\ensuremath{\mathtt{eff}^{+}(#1)}}
\newcommand{\effNegExt}[1]{\ensuremath{\mathtt{eff}^{-}(#1)}}
\newcommand{\preCondExt}[1]{\ensuremath{\mathtt{prec}(#1)}}
\newcommand{\parOrd}{\ensuremath{\mathord{\prec}}}
\newcommand{\seqSet}[1]{\ensuremath{\mathcal{L}(#1)}}

% math notations
\newcommand{\NP}{\ensuremath{\mathbb{NP}}}
\newcommand{\PTIME}{\ensuremath{\mathbb{P}}}
\newcommand{\EXPTIME}{\ensuremath{\mathbb{EXPTIME}}}
\newcommand{\powerSet}[1]{\ensuremath{2^{#1}}}

% preconditions and effects
\newcommand{\Add} {\ensuremath{\mathit{add}}}
\newcommand{\Del} {\ensuremath{\mathit{del}}}
\newcommand{\PreS} {\ensuremath{\mathit{pre^{*}}}}
\newcommand{\AddS} {\ensuremath{\mathit{add^{*}}}}
\newcommand{\DelS} {\ensuremath{\mathit{del^{*}}}}
\newcommand{\singlePrec} {\ensuremath{\mathit{ \mathord{\prec} }}}
\newcommand{\tasks} {\ensuremath{\mathit{tasks}}}
\newcommand{\Eff} {\ensuremath{\mathit{eff}}}  % example command without arguments
\newcommand{\Pre} {\ensuremath{\mathit{pre}}}  % (again)
\newcommand{\PossEffPlus} {\ensuremath{\mathit{\textit{poss-eff}^{+}_{*}}}}
\newcommand{\PossEffMinus} {\ensuremath{\mathit{\textit{poss-eff}^{-}_{*}}}}
\newcommand{\RelPossEffPlus} {\ensuremath{\mathit{\textit{poss-eff}^{\emptyset +}_{*}}}}
\newcommand{\RelPossEffMinus} {\ensuremath{\mathit{\textit{poss-eff}^{\emptyset -}_{*}}}}

% action definitions for self-written package 'aiplans'
\scheme{SinglePrec}{0}{
  text = {\textbf{A}},
  pre  = {a}
}
\scheme{SingleAdd}{0}{
  text = {\textbf{B}},
  effs = { a}
}
\scheme{SingleDelete}{0}{
  text = {\textbf{C}},
  effs = { {{$\neg$}a} }
}
\scheme{SinglePrecAlt}{0}{
  text = {\textbf{D}},
  pre  = {a}
}
\scheme{SingleAddAlt}{0}{
  text = {\textbf{E}},
  effs = { a}
}
\scheme{PrimitiveA}{0}{
  text = {\textbf{A}},
  pre  = {a},
  effs = { {{$\neg$}a}, {b} }
}
\scheme{PrimitiveB}{0}{
  text = {\textbf{B}},
  pre = {b},
  effs = {{{$\neg$}b}, {c}}
}
\scheme{PrimitiveC}{0}{
  text = {\textbf{C}},
  pre = {c},
  eff = {{$\neg$}c}
}


\Psays{Ying, there are several unresolved references, can you please fix all of them? \includegraphics[width=\textwidth]{errors}}

\Psays{There are quote a range of 'typical' errors throughout, see the list below}
\begin{itemize}
  \item Your quotes are wrong. You have to always use `6/9', ``66/99'', not '9/9'
  \item Most citations are wrong. See my template for the rules.
  \item You cite a paper by Erol twice, please fix (and remove the duplicate reference)
  \item Gregor's AAAI'22 is missing page numbers.
  \item Sometimes you have a dot that's not ending a sentence, yet you don't protect the next whitespace (see my template for an explanation)
\end{itemize}

\section{Introduction}

Hierarchical Task Network (HTN) planning is a hierarchical approach to planning. As defined by \cite{HTNSurvey},
tasks in HTN planning are either primitive, corresponding to an action that can be taken, or compound. HTN problems have a set of methods that specify how one might achieve a given compound task, by decomposing it into a set of sub-tasks. A compound task may even decompose into itself, either directly via a method, or indirectly via a sequence of method applications. If decomposition leads to a sequence of primitive tasks executable from the initial state, then this sequence of actions is a solution to the problem, also known as a \emph{plan}.

In totally ordered HTN planning, or TOHTN planning, methods specify a total order on the sub-tasks. In partially ordered HTN planning, or POHTN planning, methods might only specify a partial order on the sub-tasks.

Many problems might be naturally more suited to being modelled as POHTN problems. However, for \emph{solving} the problem, it is desirable to have additional constraints that reduces the search space, while still preserving at least some of the actual solutions. For example, \cite{ErolHTNExpressivity} proved that POHTN planning is expressive enough to model undecidable problems, such as the language intersection problem of two context-free languages. On the other hand, TOHTN planning is not expressive enough to model undecideable problems. Where this is possible, it results in lower worst-case solving time. Converting the problem to a TOHTN problem allows us to exploit that fact to solve the problem more quickly.

Another benefit of transforming a POHTN problem to a TOHTN problem is that the additional structure to the problem could allow us to deploy specialised algorithms and heuristics for the totally ordered case.
The drawback to this approach is that, due to the greater expressivity of POHTN planning, there may exist POHTN problems that cannot be solved when converted to a TOHTN problem.
Fortunately, not every POHTN problem is guaranteed to be undecidable, and so could still be transformed while preserving at least one solution.
In this paper we present and investigate an algorithm for converting POHTN problems to TOHTN problems. We prove that when certain criteria are met, it guarantees that at least one solution will be preserved.
Also, we obtain a new class of decidable problems, namely those that satisfy the above mentioned criterion. Finally, we show that, even when these criteria are not met, very few problems are rendered unsolvable by the transformation, and that it greatly reduces solving time for problems, with gains being bigger for more difficult problems.

\input{02--formalism}
We now introduce how to transform a general (i.e., partially ordered) grounded HTN planning problem into a totally ordered one. Given a grounded HTN planning problem, our goal is to linearize every method's task network by adding ordering constraints to them. In particular, in order to decide which ordering constraints are to be added to a task network, we want to \emph{infer} the precondition and effects of each compound task in the task network from all primitive tasks into which it can be decomposed. 

The reason for doing such inference is that our intent is to let every linearized task network be able to be decomposed into at least one executable action sequence (though it is not always possible), and hence, we want to capture the state transition information about the action sequences that can be obtained from the task network. % which can be loosely acquired by inferring the preconditions and the effects of the compound tasks. 

Thus, generally speaking, our linearization approach consists of two phases. The first one is to infer preconditions and effects of compound tasks, and the second one is to add ordering constraints to each method's task network according to those inferred preconditions and effects. We now elaborate these two phases in detail as follows.

\subsection{Inferring Preconditions and Effects}
\section{Linearization of Methods Using Any Set of Inferred Preconditions and Effects}

From these inferred preconditions and effects $\PreS(t), \AddS(t), \DelS(t)$, (which may have been derived using the ground or lifted variant), we can run Algorithm~\ref{alg:Algorithm1} to get an linearized problem.
For an visual demonstration of what this algorithm is doing, see Figure~\ref{fig:OrderingRules} and \ref{fig:CycleBreaking}.

 \begin{figure}[h]
 	\caption{Example of Desired Additional Orderings}
 	\label{fig:OrderingRules}
	\scalebox{0.75}{
		\begin{subfigure}{4.5cm}
			\begin{tikzpicture}[auto,node distance=1.2cm,
			thick,main node/.style={circle,draw,font=\sffamily\Large\bfseries},
			square effect/.style={rectangle,draw,font=\sffamily\Large\bfseries}]



			\action{first}{SingleAdd,
				width  = 2.5em,
				height = 2.5em,
				pre length = 1em,
				eff length = 1em
			};

			\action{second}{SinglePrec,
				width  = 2.5em,
				height = 2.5em,
				pre length = 1em,
				eff length = 1em,
				body = {right = 4em of first}
			};

			\path[every node/.style={font=\sffamily\small}]
			(first) edge [->,bend right] node [left]  {} (second);
			\end{tikzpicture}

			\caption{B adds a fact in preconditions of A -- so B before A}



		\end{subfigure}
	}
	\scalebox{0.75}{
		\begin{subfigure}{4.5cm}
			\begin{tikzpicture}[auto,node distance=1.2cm,
			thick,main node/.style={circle,draw,font=\sffamily\Large\bfseries},
			square effect/.style={rectangle,draw,font=\sffamily\Large\bfseries}]

			\action{A}{SingleDelete,
				width  = 2.5em,
				height = 2.5em,
				pre length = 1em,
				eff length = 1em
				%body = {below left = 2em and 2em of }
			};

			\action{B}{SinglePrec,
				width  = 2.5em,
				height = 2.5em,
				pre length = 1em,
				eff length = 1em,
				body = {left = 4em of A}
			};

			\path[every node/.style={font=\sffamily\small}]
			(B) edge [->,bend right] node [left]  {} (A);
			\end{tikzpicture}
			\caption{C deletes a fact in preconditions of A  -- so A before C}
		\end{subfigure}
	}

	\scalebox{0.75}{
		\begin{subfigure}{4.5cm}
			\begin{tikzpicture}[auto,node distance=1.2cm,
			thick,main node/.style={circle,draw,font=\sffamily\Large\bfseries},
			square effect/.style={rectangle,draw,font=\sffamily\Large\bfseries}]

			\action{A}{SingleAdd,
				width  = 2.5em,
				height = 2.5em,
				pre length = 1em,
				eff length = 1em
			};

			\action{B}{SingleDelete,
				width  = 2.5em,
				height = 2.5em,
				pre length = 1em,
				eff length = 1em,
				body = {left = 4em of A}
			};

			\path[every node/.style={font=\sffamily\small}]
			(B) edge [->, bend right] node [left]  {} (A);
			\end{tikzpicture}
			\caption{B adds a fact that C deletes -- so C before B}
		\end{subfigure}
	}
\end{figure}


\begin{figure}[h]
	\caption{Example of Cycle-Breaking the Desired Additional Orderings}
	\label{fig:CycleBreaking}
	\scalebox{0.75}{
		\begin{subfigure}{7cm}
		 	\centering
			\caption{Resulting method ordering has a cycle}
			\label{fig:CycleBreaking1}
			\begin{tikzpicture}[auto,node distance=1.2cm,
			thick,main node/.style={circle,draw,font=\sffamily\Large\bfseries},
			square effect/.style={rectangle,draw,font=\sffamily\Large\bfseries}]


			\action{B}{SingleAdd,
				width  = 2.5em,
				height = 2.5em,
				pre length = 1em,
				eff length = 1em
			};

			\action{A}{SinglePrec,
				width  = 2.5em,
				height = 2.5em,
				pre length = 1em,
				eff length = 1em,
				body = {right = 4em of first}
			};

			\action{C}{SingleDelete,
				width  = 2.5em,
				height = 2.5em,
				pre length = 1em,
				eff length = 1em,
				body = {right = 4em of second}
			};

			\path[every node/.style={font=\sffamily\small}]
			(B) edge [->,bend right] node [left]  {} (A)
			(A) edge [->,bend right] node [left]  {} (C)
			(C) edge [->,bend right] node [left]  {} (B);
			\end{tikzpicture}


		\end{subfigure}
	}

	\scalebox{0.75}{
		\begin{subfigure}{6cm}
		 %	\centering
			\caption{Delete any random ordering to break it}
			\label{fig:CycleBreaking2}
			\begin{tikzpicture}[auto,node distance=1.2cm,
			thick,main node/.style={circle,draw,font=\sffamily\Large\bfseries},
			square effect/.style={rectangle,draw,font=\sffamily\Large\bfseries}]


			\action{B}{SingleAdd,
				width  = 2.5em,
				height = 2.5em,
				pre length = 1em,
				eff length = 1em
			};

			\action{A}{SinglePrec,
				width  = 2.5em,
				height = 2.5em,
				pre length = 1em,
				eff length = 1em,
				body = {right = 4em of first}
			};

			\action{C}{SingleDelete,
				width  = 2.5em,
				height = 2.5em,
				pre length = 1em,
				eff length = 1em,
				body = {right = 4em of second}
			};

			\path[every node/.style={font=\sffamily\small}]
			(B) edge [->,bend right] node [left]  {} (A)
			(C) edge [->,bend right] node [left]  {} (B);
			\end{tikzpicture}

		\end{subfigure}
	}
\end{figure}

\section{Theoretical Properties of Linearization Result}

We now discuss the implications of the above-described linearization procedure, starting by the runtime of the procedure.

\paragraph{Runtime} The procedure runs in polynomial time, independent of which inference procedure is used for inferring compound tasks' preconditions and effects. Concretely, given a POHTN planning problem, inferring the precondition and effects of \emph{one} compound tasks using the simple inference takes time at most $\mathcal{O}(|\methods{}| \times M)$ with
\begin{displaymath}
	M =  \max_{(c, (T, \parOrd{}, \alpha)) \in \methods{}} |T|
\end{displaymath}
% that is, $M$ is the number of tasks in the largest task network among all methods. 
This upper bound holds because the inductive inference procedure visits all methods at the worst case, and each method contains at most $M$ subtasks. In fact, one can also observe that this number is also the upper bound for the runtime of inferring \emph{all} compound tasks' preconditions and effects. This is because after visiting all methods, the inductive procedure has naturally done processing (inferring) all compound tasks. Further, the complex inference procedure by \shortCite{ConnyPreEstimation} also has polynomial time complexity, as shown in their original work.

The task of adding ordering constraints to a method $m = (c, (T, \parOrd{}, \alpha))$ takes time $\mathcal{O}(|T|^{2})$ because we need to check, for \emph{every} pair of $t, t' \in T$, whether an ordering constraint over them should be added. Additionally, since at most $\mathcal{O}(|T|^{2})$ ordering constraints will be inserted, the number of ordering constraints that will be removed for cycle-breaking will not exceed this number. Taken together, linearizing a POHTN planning problem can be done in polynomial time.

\begin{theorem}\label{thm:Runtime}
	Transforming a POHTN planning problem into a totally ordered one takes polynomial time, independent of which inference procedure is used.
\end{theorem}
% \begin{proof}
% 	Let $|T|$ = $|T_P| + |T_C|$.

% 	To calculate $\PreS, \AddS, \DelS$ for each task $t$. We explore it's compound sub-tasks until we collect all primitive tasks. Then loop over $\PreS, \AddS, \DelS$ for the root $t$, over each fact, over each primitive task in the sub-tree. This is obviously polynomial in terms of $|T|$.

% 	Lines~\ref{alg:StartBuildGraph}-\ref{alg:EndBuildGraph} of Algorithm~\ref{alg:Algorithm1} iterates over every method, which iterates over every fact, which iterates over every sub-task in that method, which iterates over every other sub-task in that method. Leading to at most $\mathcal{O}(|M| * |F| * |T|^2)$, since number of sub-tasks per method is upper bounded by $T$.

% 	Cycle-breaking, done by lines \ref{alg:StartCycleBreak} to \ref{alg:EndCycleBreak} can be done via depth-first search per cycle to break, which is also polynomial to $|T|$.

% 	Line~\ref{alg:TopSort} can be done via topological sort of a graph where sub-tasks are nodes, and orderings are edges, which is known to be in $\mathcal{O}(V+E)$ in time.
% 	The number of sub-tasks and edges are upper bounded by $T$, so approximately $\mathcal{O}(|M| * |T|)$.

% 	So the algorithm takes polynomial time.
% \end{proof}

\paragraph{Soundness} More importantly, the presented linearization approach ensures that any solution to the TOHTN planning problem $\Pi'$ obtained by linearizing a partially ordered one $\Pi$ is also a solution to $\Pi$. That is, the solution set of $\Pi'$ is a \emph{subset} of that of $\Pi$. We can view this property as a certificate for the \emph{soundness} of our linearizing approach.

\begin{theorem}
	Let $\Pi$ be a POHTN planning problem, $\Pi'$ a TOHTN planning problem obtained from $\Pi$ by using the linearization approach, and $\mathcal{S}(\Pi)$ and $\mathcal{S}(\Pi')$ the set of all solutions to $\Pi$ and to $\Pi'$, respectively. Then, $\mathcal{S}(\Pi') \subseteq \mathcal{S}(\Pi)$. 
	\label{thm:Soundness}
\end{theorem}

\begin{proof}
	The new desired orderings for a method include all of the orderings already required by the method originally. The algorithm then turns the tasks and new desired orderings between them into a directed graph, and the new ordering is produced by performing a topological sort on the nodes of that graph. This means we do not modify the sub-tasks a method produces, just the ordering between them, so the set of plans from the totally ordered method is just a subset of the plans possible from the partially ordered one. Any solution to the linearized problem is then obviously a solution to the original problem.
\end{proof}

\begin{theorem}\label{thm:notCompleteness}
	Given a POHTN planning problem $P$ and TOHTN problem
	$P'$ obtained from $P$ by using Algorithm~\ref{alg:Algorithm1}
	then $sol(P')$ may be empty.
\end{theorem}
\begin{proof}
	This algorithm linearizes all the methods to be totally ordered. Since sub-tasks inherit the orderings of their parents, it's impossible to preserve a solution that requires the interleaving of sub-tasks if their respective parents are already ordered with respect to each other. This proves that the algorithm can remove some possible action sequences, assuming the original domain was not already totally ordered. Consider the simple example problem:
	The only decomposition for this problem results in the set of un-ordered actions $\{A, B, C\}$.
	If we consider that for the $3!$ linearizations of this set, the only executable one is $A, B, C$. This is impossible to achieve by linearized methods, since ordering either AB before C or C before AB will exclude the solution.
	Even if we were to produce $k!$ methods for each partially ordered method with $k$ unordered sub-tasks, we would not be able to preserve any solution for this problem.
	This proves that the algorithm can remove all solutions, so Algorithm~\ref{alg:Algorithm1} is not complete. Note that incompleteness already follows from complexity theory as it's theoretically impossible to turn an arbitrary undecidable problem into a decidable one.
	Specifically, solutions that require interleaving of sub-tasks will not be preserved, as the example above demonstrates.
\end{proof}



\begin{description}
	\item[Linearization Criteria] If the linearization process does not find any desired ordering contradicting the existing ones for some method, or in other words, \emph{cycle-breaking} did not occur, that method is said to meet the linearization criteria. If all methods in a problem meet the linearization criteria, then the problem meets the linearization criteria as well.
\end{description}



\begin{theorem} \label{thm:RelaxedPreconditions}
	Given a POHTN planning problem $P$ and TOHTN problem $P'$ obtained from $P$ by using Algorithm~\ref{alg:Algorithm1}, \emph{only if $P$, the entire problem meets the linearization criteria},
	then we know executability-relaxed preconditions for any $c$ are indeed the only ones needed to execute any refinement of $c$.
\end{theorem}
\begin{proof}

	For some method with at least two sub-tasks, $a$ and $b$,
	the desired orderings are to put $a$ with $f \in \Add(a)$ before $b$ with $f \in \Pre(b)$. We're assuming no cycle-breaking, so this was actually achieved. If such a precondition can't be met this way, (such an $a$ does not exist, or the method already enforces that $b$ executes before $a$) by definition it must be in the executability-relaxed preconditions calculated.

	Therefore, any precondition of $c$'s children that can't be met by placing other children of $c$ before it is part - so it is indeed a super-set of actual preconditions, assuming no cycle-breaking was needed.

	If even one partially ordered method did need cycle-breaking, than of course it means nothing - we may indeed need other preconditions to make refinements of $c$ executable. Then those results no longer hold.
\end{proof}

We use Theorem~\ref{thm:RelaxedPreconditions} as support for the following main result. We will now prove that our algorithm preserves at least one solution as long as Algorithm~\ref{alg:Algorithm1} uses simple inference or preconditions and possible effects from complex inference, and this criterion is met.

\begin{theorem}\label{thm:SpecialCase}
	Given a POHTN planning problem $P$ and TOHTN problem $P'$ obtained from $P$ by using Algorithm~\ref{alg:Algorithm1}, if $P$ meets the linearization criteria and $sol(P)$ is non-empty, then $sol(P')$ will be non-empty as well.
\end{theorem}
\begin{proof}
	Assume that there exists a solution in the PO domain. By using the same decomposition sequence in the linearized domain, we can produce the same set of actions as in the PO solution, but with a linearization of the actions decided by the linearized domain. Assume this sequence is $(a_0, a_1, ..., a_n)$. We then prove by induction over the sequence $(a_0, a_1, ..., a_n)$ that it is executable.
	If $(a_0, a_1, ..., a_n)$ is not executable, that means there exists some action $a_k,  0 < k < n$ that is not executable in the corresponding state. The action $a_k$ could only be non-executable, if one or more of its preconditions was not met. Assume one of these unmet preconditions is for existence of the state variable $A$.
	The action $a_k$ must be executable in some linearization of $\{a_0, ..., a_n\}$, as we assumed it was a PO solution. So there must exist an action $a_i$, $0 < i < n$, that will add A. Actions $a_0$ and $a_k$ must have a shared parent p in a Task Decomposition Tree. So p has subtasks $t_0$ and $t_k$ that are parents of $a_0$ and $a_k$ respectively.

	The linearization of this method would have drawn an ordering $(t_i, t_k)$ due to the way the algorithm defines $prec^{*}, add^{*}$ etc. We are assuming that all methods meet the linearization criteria, so $(t_i, t_k)$ should not be required, either directly or via inheritance from parent tasks. This safely enforces $(a_k, a_0)$ ordering in the final TO plan, meaning $a_0$ is not the first action in the resulting total order imposed by the algorithm. In other words, if $a_k$’s precondition could be met by any action $a_i$, $a_i$ would be ordered in front of it.

	If $a_i$ does not exist then $a_k$ can never be executed for any linearization of $\{a_0, ...a_n\}$, contradicting the assumption that this was a PO solution. Since each action in the solution is executable, the entire sequence is executable linearization of actions produced by decomposition of initial task, i.e.\ the solution.
\end{proof}




There are several possible levels of \emph{completeness} for a problem compilation.
\begin{itemize}
	\item all solutions remain
	\item at least one solution remains
	\item all optimal solutions remain
	\item at least one optimal solution remains
\end{itemize}
Given what's proven in Theorem~\ref{thm:notCompleteness} and \ref{thm:SpecialCase}, Algorithm~\ref{alg:Algorithm1} guarantees at least one solution remains, if linearization criteria is met. Otherwise, Algorithm~\ref{alg:Algorithm1} makes no completeness guarantees at all -- it may remove all possible solutions. Finally, Algorithm~\ref{alg:Algorithm1} makes no decisions on any metric of optimality, so obviously cannot guarantee completeness that any optimal solutions remain.


\begin{theorem}\label{thm:newClass}
	Problems $P$ that satisfy the linearization criteria form a new class of decidable problems.
\end{theorem}
\begin{proof}
	A problem $P'$ obtained from applying Algorithm~\ref{alg:Algorithm1} to any arbitrary $P$ is a totally ordered problem, which are known to be decidable, as proven by \cite{Alford2015TightHTNBounds}.
\end{proof}


\section{Empirical Evaluation}

We compare the runtime of several state-of-the-art planning systems on some transformed POHTN problems. All problems are taken from two benchmark sets prepared for the International Planning Competition (IPC) 2020. The first set \footnote{https://github.com/panda-planner-dev/ipc2020-domains/tree/master/partial-order} was used in the partially ordered track in the 2020 IPC benchmark, while the second set \footnote{https://github.com/panda-planner-dev/domains/tree/master/partial-order} went un-used for various reasons. Each problem has the standard IPC time limit of 30 minutes. Any grounding and pre-processing time needed is included as part of the 'solving' time. Since the pre-processing may \emph{potentially} remove all solutions, where the planning system can prove that the pre-processing renders a problem unsolvable, we use the remaining time to solve the original problem, called a \textit{re-run}.

\subsection{Hardware and Planning Systems}
The empirical evaluation was conducted on a machine with 30GiB of memory and 4 vCPUs, each with 2 GiB RAM. Recent work by \cite{HTN2SAS} shows that the PO panda$_\pi$, achieved similar coverage and slightly lower IPC score than the best TO planner tested. Also tested is HyperTensioN by \cite{hypertension} an TOHTN solver and the IPC 2020 winner, and Lilotane by \cite{Lilotane} another TOHTN solver and the IPC 2020 runner-up.

We test the $panda_{\pi}$ solver by \cite{useClassicalHuristicICAPS18,useClassicalHeuristicIJCAI19,progressionsearchJAIR20}, on their default configuration. This turns problem into a classical planning problem (in RC model), uses Greedy best first search, FF heuristic by \cite{FF}, and visited lists. The visited lists use task sequence hashing, visited checking with task sequences, layer hashing, order pairs hashing, and GI-complete equivalence checking for partially-ordered task networks.

We also try that configuration with FF swapped for the Add heuristic by \cite{Add}.

We compare the performance of panda$_\pi$ on the original problem and the pre-processed problem. The same setting is used for re-runs. For other planners, panda$_\pi$ with rc2(Add) is used for re-runs. We also test the latest version of TOHTN planners HyperTensioN and Lilotane on the transformed problems. Re-runs use panda$_{\pi}$ with the best heuristic(Add). We test both Complex and Simple Inference for linearization. Note that Complex Inference can only be inferred on grounded problems, but Lilotane and HyperTensioN are lifted planners.

The results table for both coverage and IPC score can be seen on page \pageref{table:GroundedSimpleIPC}--\pageref{table:GroundedComplexIPC}.
As we can see from those figures, using complex inference results in ?  increase/decrease/similar, coverage and IPC score.

The planner that has achieved the highest overall coverage is panda$_\pi$ with FF, and highest IPC score is panda$_\pi$ planner with Add heuristic, which reflects findings in previous papers.

For each partial-order planner, e.g. RC Add represents panda$_\pi$ using RC and Add heuristic,
the PO column shows it's performance on the original problem, and the TO column shows performance on the linearized problem. For each total-order planner, the column beneath it shows performance on the linearized problem.
The IPC score has a maximum of 1 per domain, for the 12 domains in total.

The best results for Simple Inference on Grounded problems, as in Table~\ref{table:GroundedSimpleCoverage} and ~\ref{table:GroundedSimpleIPC},
achieve lower coverage but higher IPC score than Complex Inference on Grounded problems, as seen in Table~\ref{table:GroundedComplexCoverage} and \ref{table:GroundedComplexIPC}. This is due to complex inference taking a much greater ~98 seconds on average, where simple inference takes ~3 seconds on average. However, the slightly higher coverage does make it appear the complex inference pays off with \emph{better} linearized domains. E.g.\ using complex inference consistently makes 7 problems solvable, whereas using simple inference, or no pre-processing, only solves for 5.

Unfortunately, when using simple inference, there are no problems for which all methods are linearized without conflict.
When using complex inference, the number of methods in all problems that are linearized without conflict is generally higher, with a few problems being provably solvable.

\input{table--GroundedSimpleCoverage}
\input{table--GroundedSimpleIPC}
\input{table--LiftedSimpleCoverage}
\begin{table}
	\centering
	\caption{IPC score achieved, using Simple Inference on Lifted Problems}
	\label{table:LiftedSimpleIPC}
	\scalebox{0.7} {
\begin{tabular}{lccccccccccl}
	\toprule
	& \multicolumn{2}{c}{RC Add} & \multicolumn{2}{c}{RC FF} & HyperTensioN & Lilotane \\
	\cmidrule(lr){2-3} \cmidrule(lr){4-5}
	&PO & TO & PO & TO  \\
\midrule
Barman-BDI & 0.07 & \textbf{0.38} & 0.07 & \textbf{0.35} & \textbf{0.85} & 0.72  \\
Monroe Fully Observ.$^{7}$ & \textbf{0.56} & 0.52 & \textbf{0.47} & 0.45 & \textbf{0.95} & 0.54  \\
Monroe Part. Observ.$^{2}$ & \textbf{0.31} & 0.28 & \textbf{0.31} & 0.29 & 0.09 & \textbf{0.63}  \\
PCP$^{18}$ & \textbf{0.83} & \textbf{0.83} & \textbf{0.82} & \textbf{0.82} & 0.82 & 0.0  \\
Rover & 0.28 & \textbf{0.61} & 0.2 & \textbf{0.56} & \textbf{0.99} & 0.97  \\
Satellite & 0.9 & \textbf{1.0} & 0.99 & \textbf{1.0} & \textbf{1.0} & \textbf{1.0}  \\
Transport & 0.24 & \textbf{0.6} & 0.27 & \textbf{0.32} & 0.61 & \textbf{0.78}  \\
UM-Translog$^{1}$ & \textbf{1.0} & 0.99 & \textbf{1.0} & \textbf{1.0} & \textbf{1.0} & 0.92  \\
Woodworking & \textbf{0.38} & 0.23 & \textbf{0.36} & 0.23 & 0.26 & 0.2  \\
\midrule
Monroe & \textbf{0.77} & 0.74 & \textbf{0.76} & 0.73 & \textbf{1.0} & 0.96  \\
SmartPhone$^{3}$ & 0.71 & \textbf{1.0} & 0.71 & \textbf{0.86} & 0.79 & 0.14  \\
Zenotravel & \textbf{1.0} & \textbf{1.0} & \textbf{1.0} & \textbf{1.0} & \textbf{1.0} & \textbf{1.0}  \\
\midrule
IPC score & 7.1 & \textbf{8.2} & 7.0 & \textbf{7.6} & \textbf{9.4} & 7.9  \\
\bottomrule
\end{tabular}
	}
\end{table}

\input{table--GroundedComplexCoverage}
\input{table--GroundedComplexIPC}
\begin{table}
	\centering
	\caption{Number of Problems per Domain for Which \emph{all} methods were linearizable, and the Average Number of Methods with more than 1 sub-task that were in that Domain}
	\label{table:FullyLinearizedProblems}
	\scalebox{0.75} {
	\begin{tabular}{lcccl}
		\toprule
		Domain      & Avg. Num. Methods ($>$ 1) & Num. Problems \\
		\midrule
		SmartPhone  & 4.0                   &  1     \\
		Transport   & 1144.9                & 40       \\
		UM-Translog & 1.1                    & 17   \\
		Woodworking & 2035              &   21    \\
				\bottomrule
	\end{tabular}
	}
\end{table}

\begin{table}
	\centering
	\caption{These results are from using Simple Inference. Table shows the average number of methods per domain with more than 1 sub-task, the average number of those methods that needed cycle-breaking, and the percentage of successfully linearized methods.}
	\label{table:SimpleLinearizable}
		\scalebox{0.65} {
	\begin{tabular}{lcccl}
		\toprule
		& Avg. Num. Methods ($>$ 1) & Cycle-breaking & Percentage &  \\
		\midrule
Barman-BDI & 4870.0 & 3717.1 & 28.6  \\
Monroe & 12127.3 & 3447.8 & 72.4  \\
Monroe Fully Observ. & 49450.6 & 15813.0 & 68.0  \\
Monroe Part. Observ. & 49470.9 & 15809.3 & 68.0  \\
PCP & 12.3 & 12.3 & 0.0  \\
Rover & 509.4 & 355.2 & 32.3  \\
Satellite & 70.6 & 36.1 & 50.9  \\
SmartPhone & 6.3 & 4.0 & 21.8  \\
Transport & 1144.9 & 1144.9 & 0.0  \\
UM-Translog & 1.5 & 1.5 & 0.0  \\
Woodworking & 1927.3 & 1927.3 & 0.0  \\
Zenotravel & 190.2 & 99.4 & 47.0  \\
		\bottomrule
	\end{tabular}
}
\end{table}

\begin{table}
	\centering
	\caption{These results are the same as Table~\ref{table:SimpleLinearizable}, but from using Complex Inference.}
	\label{table:ComplexLinearizable}
	\scalebox{0.65} {
	\begin{tabular}{lcccl}
		\toprule
		&  Avg. Num. Methods ($>$ 1)& Cycle-breaking & Percentage &  \\
		\midrule
			Barman-BDI &  4870.0 & 1836.0 & 62.2  \\
			Monroe & 12127.3 & 216.9 & 98.2  \\
			Monroe Fully Observ. & 49450.6 & 1016.2 & 97.9  \\
			Monroe Part. Observ. & 49470.9 & 1010.0 & 98.0  \\
			PCP & 12.3 & 4.9 & 59.8  \\
			Rover & 509.4 & 134.8 & 72.4  \\
			Satellite & 70.6 & 10.7 & 77.9  \\
			SmartPhone & 6.3 & 0.9 & 77.4  \\
			Transport & 1144.9 & 0.0 & 100.0  \\
			UM-Translog & 1.5 & 0.4 & 87.9  \\
			Woodworking & 1927.3 & 476.0 & 85.5  \\
			Zenotravel & 190.2 & 16.4 & 91.6  \\
			\bottomrule
		\end{tabular}
		}
\end{table}


While simple linearization can linearize almost none of the actaully partially ordered methods, with complex linearization, we can successfully linearize a fair amount of the partially-ordered methods too. (Note that this does not result in a guaranteed solvability-preserving problem unless all methods conform to the criteria though.)

\begin{table}
	\caption{For the Problems in Each Domain, Average Percentage of Partially-Ordered Methods Linearized Without Cycle-break per Problem, for each Domain. Shows Results for Both Inference Methods.}\label{table:POMethodBreak}
	\begin{tabular}{lccl}
		\toprule
		Domain & Complex & Simple &  \\
		\midrule
		Transport & 1.0 & 0.0  \\
		UM-Translog & 0.8 & 0.0  \\
		Satellite & 1.0 & 0.0  \\
		Monroe Fully Observ. & 1.0 & 0.0  \\
		Rover & 1.0 & 0.0  \\
		Woodworking & 0.9 & 0.0  \\
		Monroe Part. Observ. & 1.0 & 0.0  \\
		PCP & 0.0 & 0.0  \\
		Barman-BDI & 1.0 & 0.0  \\
		SmartPhone & 0.1 & 0.1  \\
		Monroe & 0.8 & 0.0  \\
		Zenotravel & 1.0 & 0.0  \\
		\bottomrule
	\end{tabular}
\end{table}


\section{Related Work}

There exist other planners that also compile POHTN problems to decidable planning classes e.g. the planners by \cite{HTN2STRIPS} and HTN2SAS planner by \cite{HTN2SAS} convert POHTN to classical planning problems. To the best of my knowledge, this is the first work that focuses on converting POHTN to TOHTN problems.

\cite{Erol1996HTNComplexity} proved that it is NP-complete to decide whether a partially ordered set
of actions has an executable linearization. Our criteria would work on a problem where the only method is for the initial task, and it’s sub-tasks are actions. That is essentially just ordering a partially ordered plan, so the criteria can also identify a proper subset of all partially ordered plans that can be linearized in polynomial time. It can also find a executable linearization for that subset in polynomial time.


\cite{NEBEL1994125} studied various questions regarding partially ordered plans, such as the complexity of plan existence, plan optimisation, possible or necessary truth of properties before or after plan steps, given various constraints on the types of tasks allowed.  \cite{KambhampatiModalTruth1996} studied related questions and criteria under different assumptions.
In the paper by \cite{TanGruningerPOPlanComplexity}, the authors refine \cite{NEBEL1994125}'s results and show the impact of the interaction of preconditions and effects thereby giving a more detailed picture for when finding a linearization for a partially-ordered plan is possible in polynomial time, and when it remains NP-hard.

Criteria presented in \cite{NEBEL1994125} and \cite{KambhampatiModalTruth1996} and  \cite{TanGruningerPOPlanComplexity} analyse cases where actions have singleton precondition and add lists. This is a less general case than our criteria can cover, but their criteria also correctly identifies some cases as linearizable in polynomial time that our criteria is unable to determine. Thus their results are orthogonal to ours.

The paper by \shortCite{DeleteRelaxation,Bercher2021POCLComplexities} generalised \cite{NEBEL1994125}'s as well as \cite{Erol1996HTNComplexity}'s result (deciding whether a partially ordered
set of actions possesses an executable linearization) by showing that this problem does not become easier when one is allowed to insert delete-relaxed  actions.

\section{Future Work}

The maximum satisfiability problem (MAX-SAT) is the problem of determining the maximum number of clauses, of a given Boolean formula in conjunctive normal form, that can be made true by an assignment of truth values to the variables of the formula. MAX-SAT is used over SAT as it may not be possible to satisfy all orderings implied by the inferred preconditions and effects.

The current linearization procedure considers each method in isolation, and sub-tasks have their inferred $\PreS, \AddS, \DelS$. Thus we can treat each sub-task as a \emph{action}, and encode the method the same way as a classical plan, as described by \cite{RINTANEN201245}, but with the additional constraint that \emph{actions} chosen from the method sub-tasks can only be selected once, unlike in the domain.

\section{Conclusion}

This paper finds that, at least for the POHTN problems in the IPC domain, most of them are not undecidable. In other words, it is as if the POHTN domains are compressed TO domains - this linearization separates one of them back out.

We can also see this pre-processing technique leads to large gains in time/efficiency for a number of planners, at least for the problems tested here. Furthermore, there is demonstrably much more room for improvement. Though the criteria does not fire all the time, it's producing a linearization that is almost always solvable.

This method can also be used to increase the number of TOHTN problems available for purposes like comparing TOHTN planners or grounding techniques. Suppose there were $n$ POHTN problms and $m$ TOHTN problems, and this linearization could make those $n$ POHTN problems into $k$ solvable $TOHTN$ problems. Then $k+m$ (lifted or grounded) TOHTN problems exist now. And from empirical evaluation, $k$ may be close to $n$.

Finally, in this paper we identify a subset of decidable problems within POHTN problems, and this subset is orthogonal to all others discussed in other works.


%% The file named.bst is a bibliography style file for BibTeX 0.99c
\bibliographystyle{named}
\bibliography{bib}

\end{document}

