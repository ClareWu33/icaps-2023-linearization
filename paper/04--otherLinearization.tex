\section{Linearization of Methods Using Any Set of Inferred Preconditions and Effects}

From these inferred preconditions and effects $\PreS(t), \AddS(t), \DelS(t)$, (which may have been derived using the ground or lifted variant), we can run Algorithm~\ref{alg:Algorithm1} to get an linearized problem.
For an visual demonstration of what this algorithm is doing, see Figure~\ref{fig:OrderingRules} and \ref{fig:CycleBreaking}.

 \begin{figure}[h]
 	\caption{Example of Desired Additional Orderings}
 	\label{fig:OrderingRules}
	\scalebox{0.75}{
		\begin{subfigure}{4.5cm}
			\begin{tikzpicture}[auto,node distance=1.2cm,
			thick,main node/.style={circle,draw,font=\sffamily\Large\bfseries},
			square effect/.style={rectangle,draw,font=\sffamily\Large\bfseries}]



			\action{first}{SingleAdd,
				width  = 2.5em,
				height = 2.5em,
				pre length = 1em,
				eff length = 1em
			};

			\action{second}{SinglePrec,
				width  = 2.5em,
				height = 2.5em,
				pre length = 1em,
				eff length = 1em,
				body = {right = 4em of first}
			};

			\path[every node/.style={font=\sffamily\small}]
			(first) edge [->,bend right] node [left]  {} (second);
			\end{tikzpicture}

			\caption{B adds a fact in preconditions of A -- so B before A}



		\end{subfigure}
	}
	\scalebox{0.75}{
		\begin{subfigure}{4.5cm}
			\begin{tikzpicture}[auto,node distance=1.2cm,
			thick,main node/.style={circle,draw,font=\sffamily\Large\bfseries},
			square effect/.style={rectangle,draw,font=\sffamily\Large\bfseries}]

			\action{A}{SingleDelete,
				width  = 2.5em,
				height = 2.5em,
				pre length = 1em,
				eff length = 1em
				%body = {below left = 2em and 2em of }
			};

			\action{B}{SinglePrec,
				width  = 2.5em,
				height = 2.5em,
				pre length = 1em,
				eff length = 1em,
				body = {left = 4em of A}
			};

			\path[every node/.style={font=\sffamily\small}]
			(B) edge [->,bend right] node [left]  {} (A);
			\end{tikzpicture}
			\caption{C deletes a fact in preconditions of A  -- so A before C}
		\end{subfigure}
	}

	\scalebox{0.75}{
		\begin{subfigure}{4.5cm}
			\begin{tikzpicture}[auto,node distance=1.2cm,
			thick,main node/.style={circle,draw,font=\sffamily\Large\bfseries},
			square effect/.style={rectangle,draw,font=\sffamily\Large\bfseries}]

			\action{A}{SingleAdd,
				width  = 2.5em,
				height = 2.5em,
				pre length = 1em,
				eff length = 1em
			};

			\action{B}{SingleDelete,
				width  = 2.5em,
				height = 2.5em,
				pre length = 1em,
				eff length = 1em,
				body = {left = 4em of A}
			};

			\path[every node/.style={font=\sffamily\small}]
			(B) edge [->, bend right] node [left]  {} (A);
			\end{tikzpicture}
			\caption{B adds a fact that C deletes -- so C before B}
		\end{subfigure}
	}
\end{figure}


\begin{figure}[h]
	\caption{Example of Cycle-Breaking the Desired Additional Orderings}
	\label{fig:CycleBreaking}
	\scalebox{0.75}{
		\begin{subfigure}{7cm}
		 	\centering
			\caption{Resulting method ordering has a cycle}
			\label{fig:CycleBreaking1}
			\begin{tikzpicture}[auto,node distance=1.2cm,
			thick,main node/.style={circle,draw,font=\sffamily\Large\bfseries},
			square effect/.style={rectangle,draw,font=\sffamily\Large\bfseries}]


			\action{B}{SingleAdd,
				width  = 2.5em,
				height = 2.5em,
				pre length = 1em,
				eff length = 1em
			};

			\action{A}{SinglePrec,
				width  = 2.5em,
				height = 2.5em,
				pre length = 1em,
				eff length = 1em,
				body = {right = 4em of first}
			};

			\action{C}{SingleDelete,
				width  = 2.5em,
				height = 2.5em,
				pre length = 1em,
				eff length = 1em,
				body = {right = 4em of second}
			};

			\path[every node/.style={font=\sffamily\small}]
			(B) edge [->,bend right] node [left]  {} (A)
			(A) edge [->,bend right] node [left]  {} (C)
			(C) edge [->,bend right] node [left]  {} (B);
			\end{tikzpicture}


		\end{subfigure}
	}

	\scalebox{0.75}{
		\begin{subfigure}{6cm}
		 %	\centering
			\caption{Delete any random ordering to break it}
			\label{fig:CycleBreaking2}
			\begin{tikzpicture}[auto,node distance=1.2cm,
			thick,main node/.style={circle,draw,font=\sffamily\Large\bfseries},
			square effect/.style={rectangle,draw,font=\sffamily\Large\bfseries}]


			\action{B}{SingleAdd,
				width  = 2.5em,
				height = 2.5em,
				pre length = 1em,
				eff length = 1em
			};

			\action{A}{SinglePrec,
				width  = 2.5em,
				height = 2.5em,
				pre length = 1em,
				eff length = 1em,
				body = {right = 4em of first}
			};

			\action{C}{SingleDelete,
				width  = 2.5em,
				height = 2.5em,
				pre length = 1em,
				eff length = 1em,
				body = {right = 4em of second}
			};

			\path[every node/.style={font=\sffamily\small}]
			(B) edge [->,bend right] node [left]  {} (A)
			(C) edge [->,bend right] node [left]  {} (B);
			\end{tikzpicture}

		\end{subfigure}
	}
\end{figure}
